%% For double-blind review submission, w/o CCS and ACM Reference (max submission space)
\documentclass[acmsmall,review,anonymous]{acmart}
\settopmatter{printfolios=true,printccs=false,printacmref=false}

\acmJournal{PACMPL}
\acmVolume{1}
\acmNumber{CONF} % CONF = POPL or ICFP or OOPSLA
\acmArticle{1}
\acmYear{2018}
\acmMonth{1}
\acmDOI{} % \acmDOI{10.1145/nnnnnnn.nnnnnnn}
\startPage{1}

\setcopyright{none}

%% Bibliography style
\bibliographystyle{ACM-Reference-Format}
\citestyle{acmauthoryear}   %% For author/year citations
\usepackage{pervasives}


\begin{document}

\title{Time-Sensitive Separation Logic for Hardware Design}
% \subtitle{Subtitle}
% \subtitlenote{with subtitle note}

%% Author information
%% Contents and number of authors suppressed with 'anonymous'.
%% Each author should be introduced by \author, followed by
%% \authornote (optional), \orcid (optional), \affiliation, and
%% \email.
%% An author may have multiple affiliations and/or emails; repeat the
%% appropriate command.
%% Many elements are not rendered, but should be provided for metadata
%% extraction tools.

%% Author with single affiliation.
\author{Rachit Nigam}
\affiliation{
  \institution{Cornell University}
  \country{USA}
}
\email{rnigam@cs.cornell.edu}

%% Abstract
\begin{abstract}
Write an abstract \ldots
\end{abstract}

\maketitle

\section{Introduction}

This paper describes a type system that statically proves that a circuit does not attempt to use a shared resource using multiple parallel threads of computation.
The absence of such bugs is a key requirement in proving the correctness of hardware circuits that utilize fine-grained resource sharing and reason about mutual exclusivity by reasoning about clock cycles or hardware signals.

The key idea is to use a separation logic inspired type system enriched with a notion of abstract time to enable reasoning about \emph{when} a resource is used.

\section{The Filament Language}

Filament is a core calculus for describing hardware designs that perform fine-grained resource sharing.
Filament programs consist of \emph{components} each of which can construct and schedule the execution of other components.
Computations are scheduled using \emph{time variables} which can either be concrete or abstract.
%
\nonterms{T}
%
Scheduling using concrete time variables corresponds to defining a \emph{latency-sensitive} finite state machine that controls the execution of the hardware design using the clock signal.
On the other hand, using abstract time variables is similar to designing circuits that operate using \emph{latency-insensitive} handshake signals.
The grammar of time variables forms an algebraic \emph{ring} equipped with a top element ($\infty$).
\xxx[R]{Explain why this is a ring}

Time variables are used to define intervals.
Intervals can be used to index the type of a resource, describing when it is available.
\nonterms{lf}
\nonterms{res}

\nonterms{c}

\subsection{Type Checking}
\nonterms{typ}

\paragraph{The time store}
\nonterms{sto}

\paragraph{The type store}
\nonterms{G}

\ottdefncheck

\section{Pipeline Initiation Intervals}
Pipeline parallelism is pervasive in hardware design.
Each \emph{stage} of a pipeline can execute in parallel.
In presence of fine-grained sharing of resources, the pipeline defines an \emph{initiation interval}
which is the minimum amount of time needed before a new input can be safely processed by the pipeline.
In this section, we demonstrate how Filament can be used to prove the correctness of the initiation interval of a pipeline.

The key idea is to prove that for a pipeline $M$ with initiation interval $i$, we can execute $M$ at both time steps $t$ and $t+i$.

\section{Notes}
\begin{verbatim}
- Interface for module that returns a type L when the go signal is set to high
- Interface for module that can take an abort signal and return an approximate value
- Interface for module that merges and interleaves data
\end{verbatim}


%% Acknowledgments
\begin{acks}
  This material is based upon work supported by the
  \grantsponsor{GS100000001}{National Science
    Foundation}{http://dx.doi.org/10.13039/100000001} under Grant
  No.~\grantnum{GS100000001}{nnnnnnn} and Grant
  No.~\grantnum{GS100000001}{mmmmmmm}.  Any opinions, findings, and
  conclusions or recommendations expressed in this material are those
  of the author and do not necessarily reflect the views of the
  National Science Foundation.
\end{acks}


%% Bibliography
% \bibliography{./bib/papers,./bib/venues}

\end{document}
